\def\Ponderation{XX}
\def\SigleNumero{LE SIGLE DU COURS}
\def\NomCours{Le nom du cours}
\def\DateEvaluation{La date}
\def\HeureEvaluation{L'heure de début et l'heure de fin}
\def\Session{}
\def\NomEvaluation{Le nom de l'évaluation}

% Ne rien mettre entre les accolades si l'enseignant (ou l'enseignante) est aussi le coordonnateur (ou la coordonnatrice)
\def\Coordonnatrice{}
\def\Coordonnateur{}

% Ne rien mettre entre les accolades s'il y a plus d'une section
\def\Enseignant{Enseignant}
\def\Enseignante{}

% Ne rien mettre entre les accolades s'il s'agit d'un cours à section unique
% Mettre le nom de chacun des enseignants et enseignantes entre les accolades
% Une case à cocher sera générée pour chacune des sections
\def\SectionA{}
\def\SectionB{}
\def\SectionC{}
\def\SectionD{}


\def\Directives{
\begin{enumerate}
\item Matériel autorisé:
\begin{itemize}
\item Premier item;
\item Deuxième item.
\end{itemize}
\item La liste des directives.
\item Une autre directive.
\end{enumerate}
\begin{itemize}
\item[$\rightarrow$] Une précision très importante.
\end{itemize}
}

\newcommand{\affichepoints}{}     % ← Si on veut que les # des exercices et le pointage apparaîssent
% \renewcommand{\affichepoints}{on} % ← Si on ne veut pas que les # des exercices et le pointage apparaîssent. Il faut simplement ajouter quelque chose dans l'accolade.                                     % on peut aussi changer où on commente

\documentclass{Evaluation}

\begin{document}

\pageblanche

%%%%%%%%%%%%%%%% QUESTION %%%%%%%%%%%%%%%%
\begin{questions}
\question[10] Une question de 10 points.

\vspace{10cm} % de l'espace pour répondre

\question Une question avec des parties (a), (b),...

\begin{parts}
    \part[3] Une partie de 3 points.
    \part[2] Une partie de 2 points.
\end{parts}

\pageblanche % on ajoute \pageblanche à la place de \newpage, car si on veut le format à importer dans Gradescope, on change \def\Gradescope{} dans Evaluation.cls pour la même chose mais qui contient quelque chose dans l'accolade

\question\label{questiondeuxpages} Une question avec des parties (a), (b),...et des sous-parties, sur deux pages qui contient un label pour pouvoir y faire référence plus bas

\begin{parts}
    \part[3] Une partie de 3 points.
    \vspace{15cm}
    \part
    \begin{subparts}
        \subpart[3] Une sous-partie de 3 points

\vfill % pour s'assurer que le texte soit dans le bas de la page
\pagesuivante{\ref{questiondeuxpages}}

\subpart[2] La suite sur la page suivante.

    \end{subparts}
\end{parts}

\vspace{15cm}

\question[3] Une question à choix multiple.

\begin{checkboxes}
\choice Mauvais choix
\correctchoice Bon choix
\choice Mauvais choix
\choice Mauvais choix
\end{checkboxes}

\vspace{1cm}

\question[3] Une question à choix multiple avec les choix à l'horizontal.

\begin{oneparcheckboxes}
\choice Mauvais choix
\correctchoice Bon choix
\choice Mauvais choix
\choice Mauvais choix
\end{oneparcheckboxes}
\end{questions}
\end{document} 
